%----------------------------------------------------------------------------------------
%	Debug options
%----------------------------------------------------------------------------------------
% chktex-file 2
% chktex-file 8
% chktex-file 11
% chktex-file 13
% chktex-file 18
% chktex-file 36
% chktex-file 39
% chktex-file 44
%----------------------------------------------------------------------------------------
\chapter{Introduction}\label{cha:Introduction}
% # Why does this section matter?
% # How does it reflect on investigating the gap in Scrum
% # Transition to the next section
Scrum has increasingly become the most used (61\%) \gls{framework} for \acf{asd}~\cite{Flynn20221AS} and other related fields. Despite its widespread \gls{adoption}, there is evidence of a gap between the theory and practice of Scrum, with many practitioners encountering challenges in implementing the \gls{methodology} effectively~\cite[p.~185]{Jilani2020ItG}.

The purpose of this thesis is to investigate this gap by analyzing problems and common solutions in literature and comparing them against the perceptions of Scrum practitioners. This study will provide a deeper understanding of the factors that contribute to the theory-practice gap of Scrum and how practitioners are addressing these challenges. 

The intended audience for this thesis comprises organizations that are in the process of adopting, have already adopted, or are planning to adopt the Scrum \gls{framework}. This includes individuals in various roles, such as developers, managers, and other relevant positions who have an interest in utilizing Scrum as a \gls{methodology}.

The results of this study will contribute to the body of knowledge on the theory-practice gap of Scrum and provide insights for organizations and practitioners to improve their implementation of the \gls{methodology}.

\section{Definition of the theory-practice gap of Scrum}\label{sec:DefinitionTheoryPracticeGap}
% What is a theory-practice gap
\citeA[p.~2]{Greenway2019Wia} explain that the root cause for a gap between theory and practice is commonly at the communicational level. \citeA{Carr1980TGb} adds that the jargon of the theory is too difficult or too abstract to understand. Another cause for a gap between theory and practice occurs when theoretical procedures are no longer suitable for a situation~\cite[pp.~61--76]{Carr1980TGb}.

% What is the theory-practice gap of Scrum
The theory-practice gap of Scrum refers therefore to the difference between the \glspl{principle}, values, and practices prescribed by the Scrum Guide and the way it is implemented in real-world settings.

% Reasons for a gap in Scrum
The gap exists due to a variety of factors such as inadequate training~\cite[p.~72]{Maximini2018ISi}, lack of understanding of the \gls{methodology}, cultural differences~\cite[pp.~27]{Koning2019AT}, and resistance to change~\cite[pp.~37--38]{Boehm2005Mct}.

\section{Importance of investigating the theory-practice gap of Scrum}\label{sec:ImportanceOfInvestigating}
Investigating the theory-practice gap of Scrum is important for several reasons. 

% understanding the challenges faced by the practitioners 
Firstly, due to Scrum's wide \gls{adoption}, understanding the challenges faced by the practitioners leads to possibilities to improve the \gls{methodology} of Scrum. But it is important to not only focus on the literature but to fact-check the results by actual practitioners.

% the theory-practice gap of Scrum can lead to reduced benefits
Secondly, the theory-practice gap of Scrum can lead to reduced benefits due to deviations from the recommended \glspl{method} and culture. Investigating those deviations and understanding their origins, may help future practitioners to avoid them.

% understanding the factors that contribute to the theory-practice gap of Scrum 
Thirdly, understanding the factors that contribute to the theory-practice gap of Scrum may not only help the practitioners of Scrum but the ones that want to build upon it and advance it.

Moreover, the results of this study can be used by organizations and teams to identify areas for improvement in their implementation of Scrum and to develop strategies to overcome the challenges faced in implementing the \gls{methodology} effectively.

\section{State of the Art}\label{sec:StateOfTheArt}
% Challenges
Previous studies have investigated the challenges faced by Scrum practitioners in implementing the \gls{methodology} effectively and the factors that contribute to the gap. These studies have identified a range of challenges, including removing resistance to the change~\cite[pp.~37--38]{Boehm2005Mct}, establishing the right amount of control~\cite[p.~4]{Verwijs2021Ato}, incorporating the client~\cite[p.~5]{Coyle2009Acs}, transitioning from \gls{plan-driven} roles~\cite[p.~16]{Moreira2013AtA}, scaling Scrum for larger organizations~\cite[pp.~124--126]{Maximini2018ISi}, establishing a common culture~\cite[p.~4]{Barroca2019Ata}.

% Impacts
Additionally, previous studies have explored the impact of the theory-practice gap on organizations and teams, such as decreased effectiveness due to handovers between departments~\cite[p.~119]{Vilkki2010Wai}, frustration due to frequent intervening of management~\cite[p.~29]{Koning2019AT}, incomplete \glspl{transition} to Scrum due to limited expertise and control given to Scrum Masters.

% Solutions
These studies have also provided insights into the strategies used by organizations and practitioners to overcome the challenges faced in implementing Scrum effectively. These studies have found that hiring external consultants~\cite[p.~77]{Moreira2013AtA}, intensive training of employees~\cite[p.~72]{Maximini2018ISi}, a planned approach to the \gls{transition}~\cite[p.~37]{Thorgren2019Tro} and regular analysis of the status of the \gls{transition}~\cite[p.~395]{Rubin2012ESA} help overcome the challenges faced in implementing Scrum effectively. 

The current state of the art on the theory-practice gap of Scrum highlights the importance of investigating this gap and provides a foundation for further research in this area.

\section{Overview of the research question and objectives}\label{sec:OverviewResearchQuestion}
This research aims to investigate the theory-practice gap of Scrum and to provide insights into the perceptions of Scrum practitioners. The research questions guiding this study are: 

What is the theory-practice gap of Scrum, what are its characteristics, what are the perceptions of Scrum practitioners regarding the challenges faced, what were their solutions tried and what \glspl{guideline} can be derived from them?

Therefore, the objectives of this study are to:
\begin{itemize}
    \item Provide common knowledge about Scrum and the theory-practice gap
    \item Identify the challenges faced by Scrum practitioners
    \item Investigate the impact of the theory-practice gap on organizations and teams
    \item Explore the perceptions of Scrum practitioners
    \item Provide insights for organizations and practitioners to improve their implementation of Scrum
\end{itemize}

\section{Structure of the Thesis}\label{sec:StructureThesis}
The thesis is organized into several chapters that address different aspects of the research questions and objectives. The structure of the thesis is as follows:

\begin{description}[style=nextline]
    \item[Chapter \ref{cha:Introduction}: Introduction]
    This chapter provides a general definition of the theory-practice gap of Scrum, the significance of this thesis, the research questions and objectives, and the structure of the thesis.
    \item[Chapter \ref{cha:LiteratureReview}: Literature Review]
    This chapter reviews the existing body of knowledge on Scrum, and the theory-practice gap of Scrum, including previous studies, theories and strategies to overcome the gap. It also provides the first synthesis of the findings.
    \item[Chapter \ref{cha:Method}: Method]
    This chapter outlines the research design, the methods used to collect and analyze data, and the procedures for data analysis.
    \item[Chapter \ref{cha:Discussion}: Discussion]
    This chapter presents the results of the study and discusses them in relation to the findings from the literature and ties them back to the research questions.
    \item[Chapter \ref{cha:Conclusion}: Conclusion]
    This chapter provides a summary of the main findings of the study, the conclusions drawn from the results, and the implications of the study for organizations and practitioners but it also states the limitations of the study.
    \item[Appendix]
    This appendix includes used acronyms, glossary entries, a list of participants wishing to be named, the questionnaire as well as the results and any additional information that supports the main body of the thesis.
\end{description}