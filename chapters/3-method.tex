%----------------------------------------------------------------------------------------
%	Debug options
%----------------------------------------------------------------------------------------
% chktex-file 2
% chktex-file 8
% chktex-file 11
% chktex-file 13
% chktex-file 18
% chktex-file 36
% chktex-file 39
% chktex-file 44
%----------------------------------------------------------------------------------------
\chapter{Method}\label{cha:Method}
% # Why does this section matter?
% # How does it reflect on investigating the gap in Scrum
% # Transition to the next section
This chapter outlines the processes and techniques used to collect and analyze data. It provides an overview of the research design, data collection \glspl{method}, and data analysis techniques employed in the study. The purpose of this chapter is to establish the credibility and reliability of the research by demonstrating the validity and rigor of the \glspl{method} used to collect and analyze data. In addition, it sets the foundation for the rest of the study and provides a clear understanding of the research process.

\section{Research design}\label{sec:Researchdesign}
% Why for what reason
In Chapter~\ref{cha:LiteratureReview}, various reasons and potential solutions for the theory-practice gap were identified. To ensure the validity of this research and bridge the gap between theoretical knowledge and practical application, the evaluation of practitioners is essential. The objective of the data collection is to gain a deeper understanding of the Scrum \gls{framework} in practice, to examine and expand upon the problems identified in literature. The outcome should provide insights into why individuals use Scrum, the challenges they face, and potential solutions and best practices for practitioners.

% Who is targeted
The target demographic for this study consists of individuals who have worked with, are currently working with, or have experience in the integration of the Scrum \gls{framework}. This interview aims to gather data from Software Developers, \gls{agile} coaches, managers, \ac{ux} designers, and any other individuals involved in developing or managing a product with Scrum.

\section{Data collection method}\label{sec:Datacollectionmethod}
% General: Number of questions, how many quantitative, how many open
The questionnaire employed in this study consisted of 42 questions, which were divided into 7 sections. 7 of the questions were multiple choice, 11 were range questions, and 24 were open-ended questions. The interview began with general questions regarding the participants' current role and organization. It then progressed to inquiries about specific \glspl{method} and \glspl{framework} used, followed by open-ended questions regarding the reasons for \gls{adoption}, challenges encountered, and the solutions they tried. The penultimate section comprised an evaluation part where the participants were presented with a list of reasons, challenges, and solutions, and asked to indicate the extent to which they had encountered them. The interview concluded with an open-ended question for participants to add any information that was not covered in the previous sections.

% Tool used
The interview tool, Google Forms, was selected due to its capability to provide accessible mobile support and convenient data analysis. Additionally, it also facilitated the ease of distribution and participation in the survey. The interview was conducted in German as the majority of the target organizations and communities were German-speaking.

% View questionnaire in the appendix
A complete listing of the questions posed during the interview can be located in the appendix under the label "\nameref{app:Questionnaire}." The corresponding responses provided by the interview participants can be found in the appendix under the heading "\nameref{app:QuestionnaireResults}".

% From when to when
The data for the Interview was collected from January 7th to January 25th, 2023. 

% How were interview participants selected
Invitations were distributed to various sources including educational institutions, alumni networks, professional communities on social media platforms (Xing, LinkedIn, and Facebook), as well as events hosted on gathering platforms (Meetup), and companies that market themselves as Scrum experts. 

% How many were there in the end
A total of 12 individuals participated in the study, with their anonymity making it difficult to accurately determine the number of organizations represented, however, it is estimated that a minimum of 8 organizations were represented. Among the participants, 25\% were associated with organizations involved in the development of websites or web-based products, while 16\% provided training and consulting services.

\section{Participants}\label{sec:Participants}
% Reference to the annex for the list of interview partners (6/12 were named)
A comprehensive list of the Interview participants wishing to be referenced in this thesis and suitable for reference purposes can be located in the appendix labeled "\nameref{app:ListOfParticipants}". To protect the privacy of six out of the twelve participants, who wished to stay anonymous, their email addresses were removed from the data and appendix.

\section{Data analysis methods}\label{sec:Dataanalysismethods}
The collected data was systematically processed to facilitate analysis. After the data collection was deemed sufficient and the deadline was reached, the raw data collected through Google Forms was formatted and transferred into a \href{https://github.com/mai-space/mai-joel_maximilian-bachelor_thesis/raw/main/assets/data/data-formated-for-analysis.xlsx}{spreadsheet}\footnote{The \href{https://github.com/mai-space/mai-joel_maximilian-bachelor_thesis/raw/main/assets/data/data-formated-for-analysis.xlsx}{spreadsheet} can be downloaded by clicking the word \href{https://github.com/mai-space/mai-joel_maximilian-bachelor_thesis/raw/main/assets/data/data-formated-for-analysis.xlsx}{spreadsheet} or by scanning the QR Code provided in the appendix under the chapter "\nameref{app:QRCode}"} linked with the questionnaire. The data was then divided into its original types, multiple choice, open-ended, and range questions, to enable easier processing and analysis.

The open-ended questions were especially challenging to analyze due to the wide range of values they provided. Thus, the answers were simplified into concise statements to facilitate comparison. The range questions, together with simple open questions, offered insights into the relationships between the demographic and organizational characteristics of the participants.

The multiple-choice questions were easily visualized after they were formatted, allowing for the drawing of meaningful conclusions. The results, in their final form, provide a valuable body of knowledge for comparison with the existing literature and may lead to interesting findings.