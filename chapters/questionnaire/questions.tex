%----------------------------------------------------------------------------------------
%	Debug options
%----------------------------------------------------------------------------------------
% chktex-file 1
% chktex-file 2
% chktex-file 8
% chktex-file 11
% chktex-file 13
% chktex-file 18
% chktex-file 26
% chktex-file 36
% chktex-file 39
% chktex-file 44
%----------------------------------------------------------------------------------------
\newpage
\section{Questionnaire}\label{app:Questionnaire}
\einzelantworticon = Single-choice question; \checkboxantworticon = Multiple-choice question

\section*{Über mich}
Ich heiße Joel Maximilian Mai und bin 24 Jahre alt. Zur Zeit studiere ich Medieninformatik an der Technischen Hochschule Köln. Für meine Bachelorarbeit und aus eigenem Interesse beschäftige ich mich viel mit agilen Frameworks, Methoden und ins besondere Scrum. Da ich selbst Webentwickler bin, interessiere ich mich sehr für den besten Weg Software zu entwickeln. 

\section*{Was ist der Grund für die Umfrage?}
Das Thema meiner Bachelorarbeit ist "Investigating the Theory-Practice Gap of Scrum". In der Theorie klingt Scrum sinnvoll und klar. Jedoch werden Frameworks häufig in der Theorie erweitert und für die Anwendung in der Praxis angepasst. Während meiner Literatur Recherche konnte ich viele Gründe und Lösungsansätze für die Herausforderungen aber auch für den Gap identifizieren. Jedoch, um nicht selbst einen Gap zwischen Literatur und Praxis zu erzeugen, benötige ich die Evaluierung meiner Ergebnisse durch eine Umfrage.

\section*{Welchem Zweck soll die Umfrage dienen?}
Mit Ihrer Unterstützung möchte ich mehr über den Gebrauch des Scrum Frameworks in der Praxis erfahren. Ich möchte erfahren, was gut läuft und die von mir in der Literatur identifizierten Probleme hinterfragen und ergänzen. Ich bin interessiert an den Gründen, warum Sie Scrum anwenden und an den Ursachen möglicher Herausforderungen. 

\section*{Danke für Ihre Teilnahme}
Ich danke Ihnen herzlich für die Teilnahme. Dieses Thema interessiert mich ganz besonders. Die Ergebnisse dieser Studie und der Arbeit werden hoffentlich anderen begeisterten Menschen bei ihrem Weg helfen.

\textit{Diese Umfrage dauert etwa 15 Minuten.}

\section*{Über Sie}
In diesem Abschnitt erfasse ich die zuvor genannten Rahmendaten um später in der Analyse Zusammenhänge zwischen der Position und Erfahrung und den beschriebenen Herausforderungen herzustellen.

\section*{Wie lautet ihre aktuelle Berufsbezeichnung?}
\kurzantwort

\section*{Wie lange arbeiten Sie schon in dem Bereich? (in Jahren)}
\kurzantwort

\section*{Hatten Sie bereits Erfahrung in einer anderen Rolle/Position, wenn ja, in welchen?}
\kurzantwort

\section*{Unternehmen und Team}
In diesem Abschnitt erfasse ich Rahmendaten über die Unternehmen um später Zusammenhänge zwischen der Größe, der Anzahl der Herausforderungen aber auch welche Herausforderungen besonders häufig auftreten herzustellen.

\section*{Wie groß ist das Unternehmen in dem Sie eingestellt sind aktuell?}
\einzelantwort{Kleinstunternehmen (weniger als 10 Personen)}
\einzelantwort{Kleinunternehmen (10-49 Personen)}
\einzelantwort{mittleres Unternehmen (50-249 Personen)}
\einzelantwort{Großunternehmen (mehr als 250 Personen)}

\section*{Welches Produkt bietet ihr Unternehmen an? (eigene Software, Webseiten für Kunden, ...)}
\kurzantwort

\section*{Wie groß ist Ihr aktuelles Team?}
\kurzantwort

\section*{An wie vielen Projekten arbeitet Ihr Team durchschnittlich gleichzeitig?}
\kurzantwort

\section*{Agile Frameworks, Agile Methoden}
Mir ist bewusst, dass viele Unternehmen nicht ausschließlich Scrum verwenden, daher erfasse ich in diesem Abschnitt die Abweichung von dem eigentlichen Scrum Guide.

\section*{Welche der folgenden agilen Methoden nutzen Sie?}
\einzelantwort{Scrum}
\einzelantwort{Scrumban}
\einzelantwort{Scrum-eXtreme Programming Hybrid}
\einzelantwort{Kanban}
\einzelantwort{Iterative}
\einzelantwort{Lean Thingking}
\einzelantwort{Weitere ...}

\section*{Für welches der folgenden Frameworks hat sich Ihr Unternehmen entschieden?}
\einzelantwort{Scrum (Nach Jeff Sutherland und Ken Schwaber)}
\einzelantwort{Disciplined Agile}
\einzelantwort{Scrums of Scrums (Jeff Sutherland)}
\einzelantwort{Nexus}
\einzelantwort{Scrum@Scale}
\einzelantwort{Enterprise Scrum}
\einzelantwort{Scaled Agile Framework (SAFe)}
\einzelantwort{Large Scale Scrum (LeSS)}
\einzelantwort{Agile Portfolio Management}
\einzelantwort{Weitere ...}

\section*{Wie sehr hält sich Ihr Unternehmen dabei an das Framework?}
\skalarvonbis{1}{10}{gar nicht}{strikt nach Lehrbuch}

\section*{An welchen Stellen weicht Ihr Unternehmen von der Theorie ab?}
\langantwort

\section*{Für wie gravierend halten Sie den Einfluss dieser Abweichung auf den Erfolg der Scrum Integration?}
\skalarvonbis{1}{5}{unbedeutend}{gravierend}

\section*{Hatten Sie bereits Erfahrung mit einem der anderen genannten Frameworks?}
\checkboxantwort{Scrum (Nach Jeff Sutherland und Ken Schwaber)}
\checkboxantwort{Disciplined Agile}
\checkboxantwort{Scrums of Scrums (Jeff Sutherland)}
\checkboxantwort{Nexus}
\checkboxantwort{Scrum@Scale}
\checkboxantwort{Enterprise Scrum}
\checkboxantwort{Scaled Agile Framework (SAFe)}
\checkboxantwort{Large Scale Scrum (LeSS)}
\checkboxantwort{Agile Portfolio Management}
\checkboxantwort{Weitere ...}

\section*{Welche der folgenden Methoden nutzt Ihr Unternehmen aktiv?}
\checkboxantwort{Daily Stand-Up}
\checkboxantwort{Retrospektiven}
\checkboxantwort{Sprint/Iterations Review}
\checkboxantwort{Sprint/Iterations Planning}
\checkboxantwort{Backlog Refinement}
\checkboxantwort{Digital Kanban Board}
\checkboxantwort{User Stories und oder Epics}
\checkboxantwort{Story Points}
\checkboxantwort{Planning Poker / Schätzungen}
\checkboxantwort{Häufige Releases}
\checkboxantwort{Work in Progress Limits}
\checkboxantwort{Physikalisches Kanbanboard}
\checkboxantwort{Weitere ...}

\section*{Wie lange dauert ein Sprint bei Ihnen in Wochen? (0 für keine Sprints)}
\einzelantwort{0}
\einzelantwort{1}
\einzelantwort{2}
\einzelantwort{3}
\einzelantwort{4}
\einzelantwort{Weitere ...}

\section*{Bestandsaufnahme: Scrum}
Um jetzt aber doch bessere Schlüsse ziehen zu können um den Gap von Scrum zu definieren, erfasse ich im folgenden Abschnitt explizite Daten zu der Erfahrung mit Scrum und das Erlernen des Frameworks.

\section*{Wie lange arbeitet Ihr Team bereits mit Scrum?}
\einzelantwort{Aktuell noch in der Testphase}
\einzelantwort{Kürzer als ein Jahr}
\einzelantwort{1-2 Jahre}
\einzelantwort{3-5 Jahre}
\einzelantwort{5 Jahre und mehr}

\section*{Wie lange arbeiten Sie schon mit Scrum? (in Jahren)}
\kurzantwort

\section*{Haben Sie schon einmal den Scrum Guide gelesen?}
\einzelantwort{Ja}
\einzelantwort{Teilweise}
\einzelantwort{Nein}

\section*{Wenn Sie sich bereits mit einer Veröffentlichung des Scrum Guides beschäftigt haben, mit welcher? (Jahr genügt)}
\kurzantwort

\section*{Welche Abteilungen nutzen Scrum in Ihrem Unternehmen? (Projektmanagement, HR, UX, UI, Konzeption, Entwicklung...)}
\langantwort

\section*{Wie viele Teammitglieder haben eine aktive Auseinandersetzung mit der Theorie von Scrum abgeschlossen und somit ein ausgeprägtes Verständnis?}
\kurzantwort

\section*{Welche Quelle halten Sie für Wertvoll um sich das Scrum Framework anzueignen?}
\langantwort

\section*{Wenn Sie ein Zertifakt für Scrum oder eine der Rollen besitzen, von welcher Quelle?}
\kurzantwort

\section*{Wie würden Sie die Expertise bewerten, die ein solches Zertifikat vermittelt?}
\skalarvonbis{1}{5}{Basiswissen}{Experte in dem Framework/Rolle}

\section*{Was waren Gründe oder Ursachen für ihr Unternehmen, das Framework Scrum zu wählen?}
\langantwort

\section*{Zu dem Gap in der Praxis}
Dieser Abschnitt ist wahrscheinlich der wichtigste von allen. Er bietet Ihnen die Möglichkeit den Erfolg der Integration von Scrum einzuschätzen und die wichtigsten Herausforderungen als auch Lösungsansätze sowie Tipps zu äußern. 

\section*{Wie würden Sie den Erfolg der Integration von Scrum in Ihrem Unternehmen bewerten?}
\skalarvonbis{1}{8}{Gescheitert}{Voller Erfolg}

\section*{Was sind die Herausforderungen der Integration von Scrum gewesen?}
\langantwort

\section*{Was waren Lösungsansätze mit denen Ihr Unternehmen versucht hat die Herausforderungen zu lösen?}
\langantwort

\section*{Welche Herausforderungen blieben bis heute bestehen?}
\langantwort

\section*{Welche Ratschläge würden Sie anderen Unternehmen, Teams oder Kollegen mitgeben um die Integration und Adaption von Scrum zu erleichtern?}
\langantwort

\section*{Halten Sie Scrum für ungeeignet für Ihren Anwendungszweck?}
\einzelantwort{ungeeignet}
\einzelantwort{projektabhängig}
\einzelantwort{geeignet}

\section*{Welche Rollen oder Methoden wurden von dem Alten mit in den neuen Prozess übernommen und warum? (z.B. bei Wasserfall -> Scrum, könnte das Projektmanager sein oder Anforderungsmanagement)}
\langantwort

\section*{Wer leitete Ihre Transformation/den Übergang zu Scrums Methoden? (Rolle/Position)}
\langantwort

\section*{Wie schätzen Sie die Selbstorganisation und das Selbstmanagement Ihres Teams ein?}
\skalarvonbis{1}{5}{von außen kontrolliert}{selbst organisiert}

\section*{Wie hoch ist Ihre Motivation Scrum durch ein anderes Framework zu ersetzen?}
\skalarvonbis{1}{5}{niedrig}{hoch}

\section*{Welches Framework empfehlen Sie alternativ?}
\kurzantwort

\section*{Was wären die Gründe für die Anwendung eines anderen Frameworks?}
\checkboxantwort{Jobwechsel (Neues Team arbeitet nicht agil)}
\checkboxantwort{Jobwechsel (Kein Programmierer mehr)}
\checkboxantwort{Management war im Weg}
\checkboxantwort{Aktueller Prozess ist ähnlich aber nicht agil}
\checkboxantwort{Der Versuch Agilität zu praktizieren schlug fehl}
\checkboxantwort{Die Koordination mit nicht agilen Teams war schwierig}
\checkboxantwort{Die Projekte litten unter schlechtem Design}
\checkboxantwort{Weitere ...}

\section*{Bekannte Herausforderungen und deren Lösungsansätze}
Dieser Teil ist fast so bedeutsam wie der vorherige. Hier möchte ich Ihnen zeigen welche Erkenntnisse ich durch die Literatur erlangen konnte. Ich brauch jedoch Ihre Hilfe dabei mir zu bestätigen welche der Gründe, Herausforderungen und Lösungsansätze tatsächlich in Ihrer Praxis stattfanden. Gerne dürfen Sie mir im Anschluss an die Umfrage Fragen zu den einzelnen Punkten stellen.

\section*{Welche der folgenden Gründe waren es, die zur Integration von Scrum in Ihrem Unternehmen gesorgt haben?}
\checkboxantwort{Vorhersehbarkeit, Transparenz und Sichtbarkeit}
\checkboxantwort{Frühe Risiko-Reduktion}
\checkboxantwort{Schnelleres Deployment an den Kunden oder Testumgebung}
\checkboxantwort{Schnelleres Return on Investment}
\checkboxantwort{Schnelleres Feedback von echten Nutzern}
\checkboxantwort{Kundenzufriedenheit ist verbessert}
\checkboxantwort{Reibungslosere Teamarbeit}
\checkboxantwort{Software Qualitätsverbesserung}
\checkboxantwort{Effektivitätsverbesserung}
\checkboxantwort{Mehr automatisiertes Testing}
\checkboxantwort{Produktivitätsverbesserung}
\checkboxantwort{Bessere Produkte für Kunden und Nutzer entwickeln}
\checkboxantwort{Verbessern von Reaktion auf sich verändernde Anforderungen und Prioritäten}
\checkboxantwort{Verbesserung von Einigkeit von Unternehmen und Entwicklern}
\checkboxantwort{Verbesserte Moral im Team}
\checkboxantwort{Anforderungsänderungen fallen leichter}
\checkboxantwort{Management entschied sich dafür}

\section*{Welche der folgenden Herausforderungen hatte Ihr Unternehmen bei der Integration von Scrum?}
\checkboxantwort{Es wurde nicht geprüft ob agile Software Entwicklung sinnvoll für das Unternehmen ist}
\checkboxantwort{Langsame Entscheidungsfindung. Zunächst ist nicht nur unklar wer die Entscheidungen zu treffen hat, aber auch die Informationsweitergabe erfolgt träge oder passiv.}
\checkboxantwort{Fehlendes Commitment sich ganz und gar der agilen Software Entwicklung zu verschreiben.}
\checkboxantwort{Es gibt keinen gemeinsamen Willen mehr über Scrum und agiler Software Entwicklung zu lernen}
\checkboxantwort{Die Integration von der agilen Kultur endete mit der Bekanntmachung von Scrum im Unternehmen}
\checkboxantwort{Fehlendes agiles Mindset und Kultur. Die Meinungen über die Sinnhaftigkeit eines agilen Mindsets wird nicht von allen geteilt}
\checkboxantwort{Zu viele Personen involviert. Viele verschiedene Meinungen über was Scrum ist und wie es integriert werden sollte}
\checkboxantwort{Scrum wurde von einen auf den anderen Tag etabliert und hat viel Schaden angerichtet}
\checkboxantwort{Es fehlenen erfahrene Mitarbeiter}
\checkboxantwort{Das Unternehmen legt keinen Wert darauf zu messen, Herausforderungen zu identifizieren und Lösungen für diese zu finden}
\checkboxantwort{Scrum wurde nur in der Entwicklung Integriert und der Rest des Unternehmens folgt dem Wasserfall Modell}
\checkboxantwort{Der Geschäftsführung ist es gleichgültig was das agile Mindset ist und sieht keinen Grund selbst Veränderungen im Unternehmen anzuordnen}
\checkboxantwort{Das Management will keine Kontrolle über die Teams abgeben und greift in die Arbeit ein}
\checkboxantwort{Das Management limitiert den Einfluss des Scrum Masters oder eines Agile Coaches}
\checkboxantwort{Das Management entschied agil zu werden ohne Abstimmung im Team}
\checkboxantwort{Das Management mikro-managet das Entwickler-Team}
\checkboxantwort{Der Scrum Master hat nicht die Qualifikation um bei der Integration von Scrum im Unternehmen zu helfen}
\checkboxantwort{Kollaboration und Koordination. Das Team ist verteilt (remote) und Austausch findet zu selten statt.}
\checkboxantwort{Das Entwickler Team greift nicht auf die Expertise ihrer Kollegen zurück}
\checkboxantwort{Das Team teilt nicht die für Scrum notwendige intrinsische Motivation}
\checkboxantwort{Das Team hat nicht die Erlaubnis entscheidungen selbst zu treffen}
\checkboxantwort{Das Team hat nicht das selbe Ziel wie das Unternehmen}
\checkboxantwort{Das Team kennt sich nicht gut genug}
\checkboxantwort{Das Team arbeitet nicht häufig zusammen sondern mehr parallel zueinander}
\checkboxantwort{Es wird nicht regelmäßig für die Inkremente des Produkts Kunden-/Nutzerfeedback gesammelt}
\checkboxantwort{Es wird kein Budget freigegeben für Fortbildungen zur Integration von agiler Software Entwicklung}
\checkboxantwort{Die Räumlichkeiten und Zusammenarbeit hat sich nicht verändert um Scrum-basiertes Arbeiten zu ermöglichen}
\checkboxantwort{Informationen über den Projektstatus werden nicht regelmäßig kommuniziert}
\checkboxantwort{Die Backlogs wird nicht konsequent und effektiv priorisiert}
\checkboxantwort{Schlüsselpositionen im Unternehmen sind nicht von Personen besetzt die die notwendige Qualifikation besitzen}
\checkboxantwort{Alte Positionen wurden nicht durch die neuen Rollen und Weiterbildungen ersetzt}
\checkboxantwort{Die Anzahl der Meetings reduzierte die Produktivität}
\checkboxantwort{Es gab kein Upfront-Design / Konzeption und oder blieb bei einem niederen Design}
\checkboxantwort{Entwickler die später am Projekt arbeiten, sind nicht Teil des Teams welches die Planung und Konzeption vollzieht}
\checkboxantwort{Erfolg des Projekts wird nicht angemessen für ein agiles Projekt gemessen und fehlinterpretiert}
\checkboxantwort{Kunden und Nutzer werden nicht die Entwicklung mit einbezogen}
\checkboxantwort{Kunden verstehen den Sinn von Konzeption und Anforderungsermittlung}
\checkboxantwort{Der Kunde hält agile Software Entwicklung für riskanter und versteht nicht die Vorteile}

\section*{Welche der folgenden Lösungsansätze hatte auch Ihr Unternehmen versucht?}
\checkboxantwort{Scrum nach Lehrbuch für bestimmte Zeit durchführen}
\checkboxantwort{Geschäftsführung um Rückendeckung für die Integration und Durchführung bitten und schriftlich festhalten}
\checkboxantwort{Training in agile und Scrum für alle Beteiligten}
\checkboxantwort{Training in Feedback- und Kommunikations-Kulur für alle Beteiligten}
\checkboxantwort{Enhancement-Sprints für Low-Prio Backlog Items}
\checkboxantwort{Concept Sprint(s) mit Kunden vor den eigentlichen Development-Sprints}
\checkboxantwort{Team entscheidet welches Framework/Methodologie geeignet ist}
\checkboxantwort{Kunde / Produkt Owener + Business Value (Estimation) entscheiden prio}
\checkboxantwort{Inspect-Adapt Zyklus(Retrospektiven) für die Scrum Integration wird kontinuierlich fortgeführt}
\checkboxantwort{Es wird nicht länger in Abteilungen gedacht sondern in Projektteams und Rollen}
\checkboxantwort{Keine Demos, Kunden nutzen/testen das aktuelle Produkt/Release}
\checkboxantwort{Mittleres Management lernt neue Aufgaben kennen und gibt Kontrolle ab}
\checkboxantwort{Jemand mit Erfahrung und umfassendem Wissen leitet die Transformation und begleitet diese}
\checkboxantwort{Daily Stand-Ups dienen zur Koordination für den kommenden Tag und es werden Probleme beseitigt die dem bestmöglichstem Tag im Weg stehen}
\checkboxantwort{Teams werden nach dem Dreyfus-Squared Model gepaart um bestmöglichste Ergebnisse zu erzielen}
\checkboxantwort{Der aktuelle Projekt Status wird vom Teamleiter im Daily-Standup kurz am Anfang präsentiert}
\checkboxantwort{Die Geschäftsführung kommuniziert strategische und operative Ziele transparent}
\checkboxantwort{Teammitglieder die an zusammenhängenden Teilen eines Produkts arbeiten, arbeiten im engen Austausch zusammen}
\checkboxantwort{Jeder Sprint sollte ein MVP als Ergebnis haben, welches von Kunden und Nutzern getestet werden kann}
\checkboxantwort{Nicht Abteilungen sitzen beisammen, sondern Projektteams}
\checkboxantwort{Nach einem Konzept-Workshop wird ein Design-Konzept Sprint durchgeführt welcher das Konzept in erste Prototypen überführt und zur Weiterentwicklung des Produkts beiträgt}
\checkboxantwort{Im besten Fall ist ein Kunde anwesend um das Produkt unter Beobachtung zu testen, um schnell Anpassungen an den Anforderungen vornehmen zu könnnen, substitutiv kann auch Remote getestet werden}
\checkboxantwort{Die Kunden wurden über die Vorteile von Scrum und agiler Software Entwicklung zu Beginn der Vertragsverhandlungen in Kenntnis gesetzt und eventuell Beispiele genannt bekommen}

\section*{Welche Folgen von agiler Transformation treffen auf Ihr Unternehmen zu?}
\checkboxantwort{Verbesserte Vorhersehbarkeit, Transparenz und Sichtbarkeit}
\checkboxantwort{Frühe Risiko-Reduktion}
\checkboxantwort{Schnelleres Deployment an den Kunden oder Testumgebung}
\checkboxantwort{Schnelleres Return on Investment}
\checkboxantwort{Schnelleres Feedback von echten Nutzern}
\checkboxantwort{Kundenzufriedenheit ist verbessert}
\checkboxantwort{Reibungslosere Teamarbeit}
\checkboxantwort{Software Qualitätsverbesserung}
\checkboxantwort{Effektivitätsverbesserung}
\checkboxantwort{Mehr automatisiertes Testing}
\checkboxantwort{Produktivitätsverbesserung}
\checkboxantwort{Bessere Produkte für Kunden und Nutzer entwickeln}
\checkboxantwort{Verbessern von Reaktion auf sich verändernde Anforderungen und Prioritäten}
\checkboxantwort{Verbesserung von Einigkeit von Unternehmen und Entwicklern}
\checkboxantwort{Verbesserte Moral im Team}
\checkboxantwort{Anforderungsänderungen fallen leichter}

\section*{Ihre Anmerkungen}
Wenn Sie das Gefühl haben, dass Sie gerne noch mehr zu dem Thema sagen möchten haben Sie hier die 
Gelegenheit dazu...

\section*{Haben Sie noch andere Gründe für den Theorie-Praxis Gap von Scrum?}
\langantwort

\section*{Wollen Sie die Quellen für diese Umfrage sehen?}
\einzelantwort{Ja}
\einzelantwort{Nein}

\section*{Möchten Sie für Rückfragen zu Ihren Antworten Ihre Email-Adresse angeben?}
\kurzantwort

\section*{Möchten Sie in meiner Abschlussarbeit als Interviewpartner genannt werden? (Andernfalls bleiben Sie anonym)}
\einzelantwort{Ja}
\einzelantwort{Nein}

\section*{Für die Nennung geben Sie bitte folgende Daten an:}
Titel Vorname Nachname\newline
Aktuelle Position in den Unternehmen\newline
Aktuelles Unternehmen für das Sie arbeiten\newline
(optional) Einen Link zu Ihrem LinkedIn/Xing/... Profil\newline
\langantwort

\section*{Möchten Sie nach Veröffentlichung der Abschlussarbeit benachrichtigt werden um sich die Ergebnisse durchzulesen?}
\einzelantwort{Ja}
\einzelantwort{Nein}
