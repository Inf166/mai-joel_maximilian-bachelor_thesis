%Erzeugt ein Abbildungsverzeichnis
\listoffigures
%Fügt die Zeile "`Abbildungsverzeichnis"' als Chapter ins Inhaltsverzeichnis ein
\addcontentsline{toc}{chapter}{Abbildungsverzeichnis}
\newpage
	
%Erzeugt ein Tabellenverzeichnis
\listoftables
%Fügt die Zeile "`Tabellenverzeichnis"' als Chapter ins Inhaltsverzeichnis ein
\addcontentsline{toc}{chapter}{Tabellenverzeichnis}
\newpage

%Erzeugt ein Pseudocode
\lstlistoflistings
%Fügt die Zeile "`Pseudocode"' als Chapter ins Inhaltsverzeichnis ein
\addcontentsline{toc}{chapter}{Pseudocodeverzeichnis}
\newpage
\begin{lstlisting}[caption=Pseudocode - Ähnliche Rezepte,label={lst:SimilarRecipes}]
// Mark Recipe as cooked
if (thisRecipe.isOpen() > 300000) { // Opened for longer than 5 minutes
    if(!MY.cookedRecipes.contains(thisRecipe)) { 
        MY.cookedRecipes.add(thisRecipe);
    }
}
\end{lstlisting}
\newpage

% To change the title from References to Bibliography:
\renewcommand\refname{Literaturverzeichnis}

%Paket für ein deutsches Literaturverzeichnis

\bibliographystyle{natdin} % or try natplain or unsrtnat
\bibliography{bibliography/biblio.bib} % refers to literatur.bib

	%Fügt die Zeile "`Literaturverzeichnis"' als Chapter ins Inhaltsverzeichnis ein
	\addcontentsline{toc}{chapter}{Literaturverzeichnis}
\newpage
